% Options for packages loaded elsewhere
\PassOptionsToPackage{unicode}{hyperref}
\PassOptionsToPackage{hyphens}{url}
%
\documentclass[
]{article}
\usepackage{lmodern}
\usepackage{amssymb,amsmath}
\usepackage{ifxetex,ifluatex}
\ifnum 0\ifxetex 1\fi\ifluatex 1\fi=0 % if pdftex
  \usepackage[T1]{fontenc}
  \usepackage[utf8]{inputenc}
  \usepackage{textcomp} % provide euro and other symbols
\else % if luatex or xetex
  \usepackage{unicode-math}
  \defaultfontfeatures{Scale=MatchLowercase}
  \defaultfontfeatures[\rmfamily]{Ligatures=TeX,Scale=1}
\fi
% Use upquote if available, for straight quotes in verbatim environments
\IfFileExists{upquote.sty}{\usepackage{upquote}}{}
\IfFileExists{microtype.sty}{% use microtype if available
  \usepackage[]{microtype}
  \UseMicrotypeSet[protrusion]{basicmath} % disable protrusion for tt fonts
}{}
\makeatletter
\@ifundefined{KOMAClassName}{% if non-KOMA class
  \IfFileExists{parskip.sty}{%
    \usepackage{parskip}
  }{% else
    \setlength{\parindent}{0pt}
    \setlength{\parskip}{6pt plus 2pt minus 1pt}}
}{% if KOMA class
  \KOMAoptions{parskip=half}}
\makeatother
\usepackage{xcolor}
\IfFileExists{xurl.sty}{\usepackage{xurl}}{} % add URL line breaks if available
\IfFileExists{bookmark.sty}{\usepackage{bookmark}}{\usepackage{hyperref}}
\hypersetup{
  hidelinks,
  pdfcreator={LaTeX via pandoc}}
\urlstyle{same} % disable monospaced font for URLs
\usepackage[margin=1in]{geometry}
\usepackage{graphicx,grffile}
\makeatletter
\def\maxwidth{\ifdim\Gin@nat@width>\linewidth\linewidth\else\Gin@nat@width\fi}
\def\maxheight{\ifdim\Gin@nat@height>\textheight\textheight\else\Gin@nat@height\fi}
\makeatother
% Scale images if necessary, so that they will not overflow the page
% margins by default, and it is still possible to overwrite the defaults
% using explicit options in \includegraphics[width, height, ...]{}
\setkeys{Gin}{width=\maxwidth,height=\maxheight,keepaspectratio}
% Set default figure placement to htbp
\makeatletter
\def\fps@figure{htbp}
\makeatother
\setlength{\emergencystretch}{3em} % prevent overfull lines
\providecommand{\tightlist}{%
  \setlength{\itemsep}{0pt}\setlength{\parskip}{0pt}}
\setcounter{secnumdepth}{-\maxdimen} % remove section numbering
\usepackage{float}
\usepackage{sectsty}

\author{}
\date{\vspace{-2.5em}}

\begin{document}

\hypertarget{introduction}{%
\section{Introduction}\label{introduction}}

Homeownership is frequently a desirable goal for policy makers given the
number of positive externalities it produces, like neighborhood
stability, a greater degree of social and political involvement,
physiological and health benefits, among others (CRS, 2019). It is also
an important component of the multidimensional poverty measure, given
the impact that owning a home has on the quality of life of the
population. In Mexico, this is lacking. According to INEGI (2020), 30\%
of the Mexican population occupying a home is not an owner and either
rents or borrows the property. Home abandonment due to mortgage payment
default is also a problem; according to INEGI 6.1 million out of all
43.8 million homes in Mexico are in a state of abandonment, this is 14
out of every 100 houses (INEGI, 2021). Given all of the above, analyzing
the determinants of mortgage default increases in Mexico is of great
importance. The results of such analysis would be useful not only for
policy makers and society, but also for financial institutions that
could improve their risk management practices.

Specifically, the objective of this investigation is to identify
macroeconomic factors, at the national and state level, that affect the
delinquency rate of mortgage loans granted by commercial banks in
Mexico. The predictive model will also serve as a useful forecasting
tool. This in order to contrast results between regions and provide an
in-depth analysis of the dynamics of delinquency in the country. The
added value of this research paper relies on the fact that there is
little to no previous research on the Mexican housing credit markets,
let alone on mortgage delinquency forecast models. The regional
component is also a key differentiator given the existing remarkable
differences within regions that have been understudied in the context of
housing credit markets. Today, most of the mortgage lenders including
INFONAVIT, the public housing entity in the country, operate without a
regional focus and grant the same credit conditions among states. For
the next sections of this paper an exhaustive literature review will be
presented, followed by the specific research questions and hypotheses.

\hypertarget{literature-review}{%
\subsection{Literature Review}\label{literature-review}}

There are usually two general hypotheses used to explain why people stop
paying their mortgage. The first is called the equity theory; it implies
that people make the rational and calculated decision to stop paying
when equity falls below a negative cut off. Under this assumption, once
the value of the mortgage exceeds the value of the house, people
deliberately decide not to pay their mortgage. This first hypothesis is
the base of the literature that focuses on modeling mortgage default as
an option. In this field of the literature the only variables that are
considered relevant to explain defaults are (1) interest rates and (2)
home values. The second general hypothesis is commonly referred to as
the ability-to-pay theory; it suggests that borrowers will continue to
pay as long as they are financially capable of doing so, this is, as
long as their income is sufficient to cover the periodic payments of the
loan. Under this hypothesis, the decision to stop paying is not as
rational and calculated as before; it is influenced by external shocks
or events that trigger the difficulty of repayment. Some of these events
could be divorce, relocation, job loss, health problems, among others
(Aristei \& Gallo, 2016).

Mosso and Lopez (2020) study the economic causes of delinquency in the
securitized mortgage portfolio in Mexico. The authors use indicators in
line with the equity theory like home value and interest rates, but they
also use income shock proxies like the national unemployment rate or oil
prices that are more in line with the ability-to-pay theory. The study
finds that the 28-day CETES rate of return, the average yield rate of
the US T-bill, the unemployment rate, the Global Indicator of Economic
Activity (IGAE), the National Consumer Price Index, the Index of Prices
and Quotations of the Mexican Stock Exchange, the average oil export
price, the international reserves, the MXN to USD exchange rate, the
House Price Index and the monetary base (M1) are all good explanatory
variables for delinquency in the securitized mortgage portfolio in
Mexico. However, the scope of the investigation is limited given that it
is only at the national level. Mosso and Lopez (2020) also highlight the
importance of such studies for the design of public policies aimed at
promoting access to housing, particularly in the case of low-income and
marginalized groups in society. The authors also mention that a natural
extension of their work would be to analyze delinquency in a broader
way, that is, not restricting the analysis to only securitized
mortgages. The present study tackles both, the national level
restriction and the securitized-only mortgages limitation, contributing
to the missing literature in Mexico.

One could argue that some of the variables most commonly used in
empirical studies (i.e.~a national unemployment rate) are not good
proxies for adverse trigger events that shock mortgage tenors' income
specifically. But the truth is, most of the literature is based on
proxies. According to Tian et. al.~(2016) trigger events that affect a
borrower's ability-to-pay are very difficult to measure, so, for
example, divorce could be proxied with the national divorce rate, or
relocation with national mobility indicators. In their study, the
authors found that these proxies are as statistically significant as the
specific reported events when predicting mortgage default, thus
empirically justifying their use.

Another interesting study is that of Aristei and Gallo (2016) for the
Italian market. Their study aims to evaluate the joint impact of
sociodemographic factors, micro-level loan characteristics and
institutional variables on the probability of mortgage default and on
the intensity of arrears in payments in Italy. Regarding
sociodemographic factors, the authors find that households whose head is
young, unemployed or immigrant have a higher probability of default.
This is important given the demographic differences within Mexican
states and the possible implications on the housing credit market.

Another important line in literature has to do with the double-trigger
theory, which derives from a combination of the equity and
ability-to-pay theories. The double-trigger hypothesis suggests that it
takes a combination of both, negative equity and job loss (or other
income shock), for people to fall into default. Gerlach and Lyons (2018)
contribute to this line of literature identifying the causes of mortgage
arrears in Europe. The authors highlight the importance of the study
mentioning that arrears put at risk the stability of the financial
system, the availability of credit in the future and the general
well-being of the population, thus identifying the key drivers of the
phenomenon is essential to be able to design adequate policies to
control them. The study found that affordability issues, like
unemployment, low income, and high mortgage payments, are significant
mortgage default determinants. They also found that longer-term arrears
are more likely to happen for households that face the double-trigger
problem; that is, households with a high level of indebtedness, and that
at the same time are exposed to income volatility. Gerlach and Lyons
(2018) emphasize on the fact that mortgage arrears are a social problem,
given the additional stress, health issues and homelessness they
trigger. In the end, mortgage default has indirect effects on public
finances and economic effects in terms of lower and more volatile
aggregate consumption, lower labor mobility, and weaker bank balance
sheets.

In addition to these double-trigger findings, Linn and Lynos (2020)
follow an incipient investigation about a third trigger that has to do
with the institutional and policy frameworks that have a significant
impact on default rates. The study, carried out in five european
countries, also aimed to analyze the determinants (borrower and loan
characteristics) of mortgage defaults, finding that there are
substantial cross-country differences in default rates that cannot be
explained only by the double-trigger theory, and that actually derive
from strong institutional differences among countries, such as lending
practices or political/legal impediments. The research highlights the
importance of studying specific characteristics in the different
countries. The present study addresses a similar geographical
segmentation but within Mexico, statewide and by region.

As stated before, different types of variables are used to explain
mortgage default in the general literature; from sociodemographic, to
macroeconomic or micro-level bank specific variables. However, the
evidence in favor of macroeconomic indicators as explanatory variables
for mortgage default is found to be more robust. Dimitrios et.
al.~(2012) present a study that contrasts macroeconomic against
bank-specific variables in Greece and found that delinquency is mainly
explained by the macroeconomic variables (like GDP, unemployment and
interest rates) and by the management quality of financial institutions.

Another study, conducted by Brent et ál. (2011), examines the mortgage
delinquency rates for loans in the United States, analyzing the impact
of variables concerning: 1) loan characteristics; 2) borrower
characteristics and 3) economy and income (unemployment and house price
index, for example). Results of the model reveal that income, the type
of loan and the health of the general economy are important determinants
of the delinquency rates. This type of analysis allowed policy makers in
the United States to address mortgage loan issues, just like in the 2009
Plan after the crisis. Their findings suggest that it is important to
analyze the determinants of mortgage delinquency in order to gather
important information that will allow financial institutions to improve
their assessment methods and to implement strategies to solve or avoid
default issues. Along with this research, a regional analysis of
household loan delinquency in the United States' population was
conducted by Wadud et ál (2019). It is important to highlight that the
regional analysis of household loans is very limited; and for Mexico,
there is little to no literature with regional analysis in this context.

In the regional study, results show that the house price, unemployment,
per capita income/expenses and the interest rates have a significant,
yet different, impact on the mortgage delinquency rate of each region.
In addition, Wadud et ál (2019) analyses the role and impact of the
behavioral and emotional factors in the mortgage delinquency rates.
Their main findings reveal that there is a significant and negative
impact of the current consumer's confidence on the delinquency rates,
which can be explained by the awareness of the lenders about the
economy´s condition or informed credit decisions. On the other hand, the
expected consumer's confidence has a positive and significant impact on
the delinquency rates probably due to the overoptimism and additional
borrowings. In sum, there seems to be an opportunity to study the
consumer's behavior in each region, together with the other
macroeconomic and microeconomic factors, in order to reduce loan
delinquencies and promote better financial practices.

Lastly, with a study conducted in Latvia, Spilberg (2020) obtained
similar outcomes to the ones already presented on this literature
review. Results derived from 59 regression models showed that the most
influential factors on the delinquency rates in Latvia were the
unemployment rate, GDP, wages growth, house price index, credit history,
LTV, among other macro and microeconomic factors. Along with these
findings, one of the most significant conclusions for the present study
was that residential mortgage loans contribute in important ways to the
life quality of people who live in developed countries; this means, a
sustainable development of the housing market, which really depends on a
high quality analysis of household credit risks. In order to have this
high quality analysis, the availability of a number of models that allow
better risk management practices plays a very important role, together
with the availability of indicator forecasts.

Clearly, there are numerous studies that study the determinants of
mortgage delinquency in order to promote better risk management
practices that protect financial institutions and society. The main
results on most of the papers show that macro and microeconomic
variables (such as GDP, unemployment, house price index, LTV, income,
M1) have a significant impact on the delinquency rates studied. However,
there are multiple limitations in the empirical studies analyzed,
including a lack of more segmented analyses with regional focus and
limited scope regarding the type of loans that are taken into account.
There are also variables that are less conventional and that remain
understudied - yet could be very important - such as energy expenditures
that have shown a significant relationship with rates of mortgage
delinquency (Kauffman, 2010) and that will be considered for this study.
This and more will be taken into account in the present study that aims
to be a value-added research that contributes to the limitations of the
current literature and to provide models and tools to carry out
important credit risk analysis and forecasts.

\hypertarget{three-lines-of-research}{%
\subsection{Three lines of research}\label{three-lines-of-research}}

\hypertarget{ability-to-pay}{%
\subsubsection{Ability to pay}\label{ability-to-pay}}

\hypertarget{willingness-to-pay}{%
\subsubsection{Willingness to pay}\label{willingness-to-pay}}

\hypertarget{confidence-and-cost-indicators-for-home-purchase}{%
\subsubsection{Confidence and cost indicators for home
purchase}\label{confidence-and-cost-indicators-for-home-purchase}}

\end{document}
